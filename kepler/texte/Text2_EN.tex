\documentclass{article}
\usepackage{epsfig}
\usepackage{color}
\usepackage{amsmath}            
\usepackage[T1]{fontenc}         
\usepackage[latin1]{inputenc}    
\usepackage[german]{babel}      
\usepackage{helvet}

\selectlanguage{german}         
\oddsidemargin15pt
\evensidemargin15pt
\textheight15cm
\textwidth20cm

\begin{document}

\hyphenation{Sonnenferne}

\thispagestyle{empty}

\noindent
\sffamily
\Huge
\pagecolor{black}
\color{yellow}
ENGLISH Das zweite Keplersche Gesetz\\
\color{blue}\hrulefill\\
\color{yellow}
~\\
Die Verbindungslinie Sonne~--~Planet �berstreicht in\\
gleichen Zeitr�umen gleiche Fl�chen.\\
~\\
\color{white}
\LARGE
\sloppy
Wir markieren die Position des Planeten regelm��ig alle 3 Wochen und
zeichnen jeweils die �berstrichenen Fl�chen ein. Alle Fl�chen haben
die gleiche Gr��e. In Sonnenn�he m�ssen die Planeten schneller als in
Sonnenferne laufen, um dieselbe Fl�che auszuf�llen.\\
\begin{center}
\epsfysize=150pt
\epsffile{Text2_EN.eps}
\end{center}
\vfill
\hfill Start der Vorf�hrung mit \color{black}\colorbox{red}{F12}
\end{document}
