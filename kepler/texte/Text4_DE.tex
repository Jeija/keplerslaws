\documentclass{article}
\usepackage{epsfig}
\usepackage{color}
\usepackage{amsmath}            
\usepackage[T1]{fontenc}         
\usepackage[latin1]{inputenc}    
\usepackage[german]{babel}      
\usepackage{helvet}

\selectlanguage{german}         
\oddsidemargin15pt
\evensidemargin15pt
\textheight15cm
\textwidth20cm

\begin{document}
\hyphenation{Planetenbahnen}
\hyphenation{Planeten}
\thispagestyle{empty}

\noindent
\sffamily
\Huge
\pagecolor{black}
\color{yellow}
Die Erde als Zentrum des Sonnensystems\\
\color{blue}\hrulefill\\
\color{white}
\LARGE
\sloppy
Wenn wir die Erde in das Zentrum r�cken, wird die Beschreibung
der Planetenbahnen kompliziert. Jetzt kreist die Sonne mit allen anderen
Planeten um die Erde.
\vfill
\hfill Start der Vorf�hrung mit \color{black}\colorbox{red}{F12}
\end{document}
