\documentclass{article}
\usepackage{epsfig}
\usepackage{color}
\usepackage{amsmath}            
\usepackage[T1]{fontenc}         
\usepackage[latin1]{inputenc}    
\usepackage[german]{babel}      
\usepackage{helvet}

\selectlanguage{german}         
\oddsidemargin15pt
\evensidemargin15pt
\textheight15cm
\textwidth20cm

\begin{document}

\hyphenation{Sternhimmel}
\hyphenation{r�ckl�ufig}
\hyphenation{Richtung}
\hyphenation{Heliozentrischem}

\thispagestyle{empty}

\noindent
\sffamily
\Huge
\pagecolor{black}
\color{yellow}
ENGLISH Die Schleifen der Marsbahn\\
\color{blue}\hrulefill\\
\color{white}
\LARGE
\sloppy
Zeichnet ein Beobachter auf der Erde den Lauf des Mars
vor dem Sternhimmel auf, so ergibt sich folgendes Bild:
\begin{center}
\epsfysize=180pt
\epsffile{Text5_EN.eps}
\end{center}
Mars bewegt sich zun�chst rechtl�ufig von Westen nach Osten,
dann dreht er f�r ein paar Wochen seine Bewegungsrichtung um
und wird r�ckl�ufig. An einem zweiten Umkehrpunkt wechselt er
wieder auf West~--~Ost~-Richtung. Dieser Bewegungsablauf l�sst
sich mit Keplers Heliozentrischem Weltbild einfach erkl�ren.
\vfill
\hfill Start der Vorf�hrung mit \color{black}\colorbox{red}{F12}
\end{document}
