\documentclass{article}
\usepackage{epsfig}
\usepackage{color}
\usepackage{amsmath}            
\usepackage[T1]{fontenc}         
\usepackage[latin1]{inputenc}    
\usepackage[german]{babel}      
\usepackage{helvet}

\selectlanguage{german}         
\oddsidemargin15pt
\evensidemargin15pt
\textheight15cm
\textwidth20cm

\begin{document}
\hyphenation{Mittelpunkt}
\hyphenation{erkl�rbar}
\hyphenation{sehen}
\thispagestyle{empty}

\noindent
\sffamily
\Huge
\pagecolor{black}
\color{yellow}
ENGLISH Kepler und das neue Weltbild\\
\color{blue}\hrulefill\\
\LARGE
\color{white}
Um die Keplerschen Gesetze der Planetenbahnen kennenzulernen,\\ 
dr�cken Sie bitte eine der folgenden Tasten:
\begin{itemize}
\item \color{black}\colorbox{red}{F1}\color{white}~~ f�r das erste Gesetz �ber die Form der Bahnen 
\item \color{black}\colorbox{red}{F2}\color{white}~~ f�r das zweite Gesetz �ber die Geschwindigkeiten 
\item \color{black}\colorbox{red}{F3}\color{white}~~ f�r das dritte Gesetz �ber die Umlaufszeiten und Abst�nde
\end{itemize}
\sloppy
Die Erde ist ein Planet unter vielen. Wenn wir die Erde in
den Mittelpunkt des Geschehens r�cken, werden die Planetenbewegungen
schwer erkl�rbar. Die Auswirkungen k�nnen Sie bei den folgenden
Punkten sehen:
\begin{itemize}
\item \color{black}\colorbox{red}{F4}\color{white}~~ Alles ``kreist`` um die Erde 
\item \color{black}\colorbox{red}{F5}\color{white}~~ Die Schleifen der Marsbahn 
\end{itemize}
\vfill
\epsfysize=2ex
\epsffile{flag_DE.eps}
bitte \color{black}\colorbox{red}{F12}\color{white} dr�cken
\hfill Weiter mit
\color{black}\colorbox{red}{F1}\color{white},~ 
\color{black}\colorbox{red}{F2}\color{white},~ 
\color{black}\colorbox{red}{F3}\color{white},~ 
\color{black}\colorbox{red}{F4}\color{white}~oder 
\color{black}\colorbox{red}{F5}\color{white}
\end{document}
