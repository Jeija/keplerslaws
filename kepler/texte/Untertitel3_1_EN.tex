\documentclass{article}
\usepackage{epsfig}
\usepackage{color}
\usepackage{amsmath}            
\usepackage[T1]{fontenc}         
\usepackage[latin1]{inputenc}    
\usepackage[german]{babel}      
\usepackage{helvet}

\selectlanguage{german}         
\oddsidemargin15pt
\evensidemargin15pt
\textheight25mm
\textwidth20cm

\begin{document}
\definecolor{bunt}{rgb}{1.0,1.0,0.9}
\thispagestyle{empty}
\sffamily
\LARGE
\pagecolor{black}
\color{white}
\noindent
EN Um die dritten Potenzen darzustellen, malen wir W�rfel mit den
gro�en Halbachsen als Seitenl�ngen. Die Rauminhalte der W�rfel
sind die dritten Potenzen der jeweiligen gro�en Halbachsen\color{bunt}.
\end{document}
