\documentclass{article}
\usepackage{epsfig}
\usepackage{color}
\usepackage{amsmath}            
\usepackage[T1]{fontenc}         
\usepackage[latin1]{inputenc}    
\usepackage[german]{babel}      
\usepackage{helvet}

\selectlanguage{german}         
\oddsidemargin15pt
\evensidemargin15pt
\textheight25mm
\textwidth20cm

\begin{document}
\hyphenation{langsameren}
\hyphenation{Sternenhimmel}
\definecolor{bunt}{rgb}{1.0,1.0,0.9}
\thispagestyle{empty}
\sffamily
\LARGE
\pagecolor{black}
\color{white}
\noindent
\sloppy
Mit der \color{red}roten Linie\color{white} ~markieren wir
jetzt die Marsposition am Sternenhimmel. Diese Bahn verl�uft 
schleifenf�rmig, da die schnellere Erde den langsameren Mars �berholt.
\end{document}
