\documentclass{article}
\usepackage{epsfig}
\usepackage{color}
\usepackage{amsmath}            
\usepackage[T1]{fontenc}         
\usepackage[latin1]{inputenc}    
\usepackage[german]{babel}      
\usepackage{helvet}

\selectlanguage{german}         
\oddsidemargin15pt
\evensidemargin15pt
\textheight15cm
\textwidth20cm

\begin{document}

\hyphenation{verhalten}
\hyphenation{Einheiten}

\thispagestyle{empty}
\noindent
\sffamily
\Huge
\pagecolor{black}
\color{yellow}
ENGLISH Das dritte Keplersche Gesetz\\
\color{blue}\hrulefill\\
\color{yellow}
~\\
Die Quadrate der Umlaufszeiten verhalten sich wie die dritten Potenzen der gro�en
Halbachsen.\\
~\\
\color{white}
\LARGE
Die Erde braucht genau 1 Jahr pro Umlauf bei einer gro�en Halbachse von
1 Einheit. Der Kleinplanet Oterma ben�tigt 8 Jahre f�r einen Umlauf.
Die Quadrate der Umlaufzeiten verhalten sich wie (1~mal~1):(8~mal~8),
d.h. wie 1:64.
Da sich die dritten Potenzen der gro�en Halbachsen
ebenso verhalten, hat die gro�e Halbachse von Oterma eine L�nge
von 4 Einheiten, denn
(1~mal~1~mal~1):(4~mal~4~mal~4) ist wieder 1:64.\\
In Formeln ausgedr�ckt:\\
\[
\frac{U_1^2}{U_2^2}=\frac{a_1^3}{a_2^3},\quad \text{hier:}
\quad\frac{1^2}{8^2}=\frac{1^3}{4^3}
\]
\vfill
\hfill Start der Vorf�hrung mit \color{black}\colorbox{red}{F12}
\end{document}
