\documentclass{article}
\usepackage{epsfig}
\usepackage{color}
\usepackage{amsmath}            
\usepackage[T1]{fontenc}         
\usepackage[latin1]{inputenc}    
\usepackage[german]{babel}      
\usepackage{helvet}

\selectlanguage{german}         
\oddsidemargin15pt
\evensidemargin15pt
\textheight15cm
\textwidth20cm

\begin{document}

\hyphenation{Kleinplaneten}
\hyphenation{Ellipse}
\hyphenation{Uranus}

\thispagestyle{empty}

\noindent
\sffamily
\Huge
\pagecolor{black}
\color{yellow}
Das erste Keplersche Gesetz\\
\color{blue}\hrulefill\\
\color{yellow}
~\\
Die Planeten bewegen sich auf Ellipsenbahnen,\\
in deren einem Brennpunkt die Sonne steht.\\
~\\
\color{white}
\LARGE
\sloppy
Alle neun Planeten (Merkur, Venus, Erde, Mars, Jupiter, Saturn,
Uranus, Neptun, Pluto) bewegen sich um die Sonne.
Dies gilt auch f�r die Kleinplaneten und Kometen (z.B. den Kometen Halley).
Die Form der Bahn ist eine Ellipse. Eine Ellipse besitzt zwei Brennpunkte.\\
Die Sonne befindet sich in einem Brennpunkt der Bahn.
\begin{center}
\epsfysize=150pt
\epsffile{Text1_DE.eps}
\end{center}
\vfill
\hfill Start der Vorf�hrung mit \color{black}\colorbox{red}{F12}
\end{document}
