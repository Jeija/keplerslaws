\documentclass{article}
\usepackage{epsfig}
\usepackage{color}
\usepackage{amsmath}            
\usepackage[T1]{fontenc}         
\usepackage[latin1]{inputenc}    
\usepackage[german]{babel}      
\usepackage{helvet}

\selectlanguage{german}         
\oddsidemargin15pt
\evensidemargin15pt
\textheight15cm
\textwidth20cm

\begin{document}

\thispagestyle{empty}

\sffamily
\Huge
\pagecolor{black}
\color{white}
\noindent
\begin{center}
Dies ist eine interaktive Demonstration der\\Planetenbewegungen.\\
~\\
~\\
\color{yellow}Starten Sie das Programm mit der Taste \color{black}\colorbox{red}{F12}\color{white}\\ 
\color{yellow}English version: press \color{black}\colorbox{red}{F11}\color{white}\\ 
\color{white}
\end{center}
~\\
~\\
\Large
Referenzen:\\
Die folgenden Bilder werden in Echtzeit auf einem PC unter Debian/GNU Linux berechnet.\\
Dieses Programm untersteht der GNU Lizenz der Free Software Foundation.\\
Es ist somit mit vollem Quellcode frei erh�ltlich, z.B. im Internet unter\\
http://www.physik.tu-muenchen.de/\verb+~+tkramer\\
\color{white}\hfill (C) Tobias Kramer 1999
\end{document}
